%%%%%%%%%%%%%%%%%%%%%%%%%%%%%%%%%%%%%%%%%
% Developer CV
% LaTeX Template
% Version 1.0 (28/1/19)
%
% This template originates from:
% http://www.LaTeXTemplates.com
%
% Authors:
% Jan Vorisek (jan@vorisek.me)
% Based on a template by Jan Küster (info@jankuester.com)
% Modified for LaTeX Templates by Vel (vel@LaTeXTemplates.com)
%
% License:
% The MIT License (see included LICENSE file)
%
%%%%%%%%%%%%%%%%%%%%%%%%%%%%%%%%%%%%%%%%%

%----------------------------------------------------------------------------------------
%   PACKAGES AND OTHER DOCUMENT CONFIGURATIONS
%----------------------------------------------------------------------------------------
\documentclass{developercv} % Default font size, values from 8-12pt are recommended
\usepackage{multicol} % header is two column
\usepackage{calc}
\usepackage{hyperref}

\addbibresource{publications.bib}

%----------------------------------------------------------------------------------------

\begin{document}

%----------------------------------------------------------------------------------------
%   TITLE AND CONTACT INFORMATION
%----------------------------------------------------------------------------------------

{ \topskip0pt \parskip0pt \setlength{\columnsep}{24pt}

\begin{multicols}{2}

	\vspace*{\fill}
	\raggedleft

	\colorbox{white}{{\HUGE\textcolor{black}{{\MakeUppercase{Jean-Philippe}}}}}

	\colorbox{black}{{\HUGE\textcolor{white}{\textbf{\MakeUppercase{Legault}}}}}

	{\huge%
		Researcher\\[2pt]
		Senior Developer}

	\columnbreak

	\vspace*{\fill}

	{%
		\raggedright
		\tt

		% minor fix for map icon
		\hspace*{1pt}\icon{MapMarker}{12}{\hspace*{2pt}Douglas, New Brunswick, Canada}
		\vspace*{5pt}

		\icon{Phone}{12}{+1 514 603 8357}
		\vspace*{5pt}

		\icon{At}{12}{\href{mailto:jeanphilippe.legault@unb.ca}{jeanphilippe.legault@unb.ca}}
		\vspace*{5pt}

		\icon{Linkedin}{12}{\href{https://www.linkedin.com/in/jean-philippe-legault-1489ab173}{jean-philippe-legault-1489ab173}}
		\vspace*{5pt}

		\icon{Github}{12}{\href{https://github.com/jeanlego}{jeanlego}}

	}

\end{multicols}
\vspace*{-\baselineskip}

{\color{linecolor}
	%%%%%%%%%% params
	\newlength{\lineWidth}
	\setlength{\lineWidth}{0.5\textwidth}
	\newlength{\topLen}
	\setlength{\topLen}{.12\textwidth}
	\newlength{\bottomLen}
	\setlength{\bottomLen}{.155\textwidth}

	%%%%%%%%%% create the lines
	\hspace*{\fill}\rule{\lineWidth}{.4pt}\hspace*{\fill}

	\vspace*{-\topLen}
	\hspace*{\fill}\rule{.4pt}{\topLen+\bottomLen}\hspace*{\fill}
	\vspace*{-\bottomLen}
}

\vspace*{-\baselineskip}
\begin{multicols}{2}

	\RTLpar\sloppy

	During my master's in Computer Science, I have gained expertise in language runtimes and FPGA development.
	My broad interests allow me to use lateral thinking to solve complex problems.
	I can efficiently visualize problems, design solutions, and communicate abstract ideas, allowing me to be an effective team member.
	My experience in academia and working as a senior developer has shaped me into an effective teacher.
	Working on both commercial and open-source projects has made me adaptable to new workflows.

	\vspace*{\fill}
	\columnbreak

	\justifying \sloppy

	\cvsect{Languages}

	\vspace*{-4pt}
	\bulletedlist{
		French---\texttt{native},
		English---\texttt{fluent}
	}

	\vspace*{5pt}

	\ifdefined\academicCV {%
		\cvsect{Research Interest}
	} \else {%
		\cvsect{Interest}
	} \fi

	\vspace*{-4pt}
	\bulletedlist{
		Software and Computer Engineering:\\Compilers,
		Distributed Computing,
		Embedded Systems,
		FPGA,
		High-Level Synthesis,
		Operating Systems,
		Virtual Machines
	}
	\vspace*{5pt}

	\cvsect{Programming Languages}

	\vspace*{-4pt}
	\bulletedlist{
		ArmV8,
		Bash,
		Blif,
		C++,
		C,
		Java,
		LaTex,
		Perl,
		Python,
		Verilog,
		VHDL
	}

	\vspace*{\fill}
\end{multicols}
}

% reset settings
\justifying

\cvsect{Education}

\begin{entrylist}

	\entry
	{2018 -- cur.}
	{Master in Computer Science}
	{University of New Brunswick | NB, Canada}
	{Finished courses and thesis, will defend on November 2020:

		\textbullet{} FPGA CAD tool (CS6999)
		\begin{smallQuote}
			Project-based: Redesign sequential logic synthesis to support multiple clock domain and multiple edge sensitivity in Odin II.
		\end{smallQuote}

		\textbullet{} Operating Systems (CS6605)
		\begin{smallQuote}
			A comprehensive study of computer operating systems: process synchronization, concurrency, multi-processor systems, virtual memory, advanced file systems, distributed systems. 
			Design an i386 multi-tasking Kernel.
		\end{smallQuote}

		\textbullet{} Machine Learning \& Data Mining (CS6735)
		\begin{smallQuote}
			Cover: inductive inference of decision trees, neural network learning, statistical learning methods, generic algorithms, Bayesian methods, Information-Theoretic classification, and reinforcement learning.
			Focus on theoretical concepts of machine learning such as inductive bias, the PAC learning framework, Occam's razor, models of noise, uniform convergence, and Fourier analysis methods.
		\end{smallQuote}

		\textbullet{} Dynamic Memory Management (CS6905)
		\begin{smallQuote}
			Study a statically typed functional language and use it to write interpreters and demonstrating these core concepts.
			Learn the main techniques of memory management including allocation, liveness detection, reference counting, compaction, and generational collectors.
		\end{smallQuote}
	}

	\shortentry
	{2019}
	{Diploma in University teaching}
	{University of New Brunswick | NB, Canada}

	\entry
	{2015 -- 2017}
	{Bachelor's in Computer Science with Honours}
	{University of New Brunswick | NB, Canada}
	{Honours in FPGA Architecture Design:

		\textbullet{} Natural Language Processing (CS4765)
		\begin{smallQuote}
			Topics include fundamental topics in natural language processing such as n-gram language models, part-of-speech tagging, parsing, lexical semantics, spelling correction, document classification.
			Students are expected to implement and evaluate a variety of natural language processing methods, as well as write reports describing their implementations and their performance.
		\end{smallQuote}

		\textbullet{} Intro to Parallel Processing (CS4745)
		\begin{smallQuote}
			Parallel computer architectures, design and analysis of parallel algorithms, parallel programming languages, case studies, selected numerical and non-numerical applications.
		\end{smallQuote}

		\textbullet{} Intro to Artificial Intelligence (CS4725)
		\begin{smallQuote}
			Introduction to intelligent agent design, problem solving using search techniques, the use of mathematical logic for knowledge representation and reasoning, decision making under uncertainty, machine learning techniques.
		\end{smallQuote}

		\textbullet{} Digital Logic Design (ECE2214)
		\begin{smallQuote}
			This is an introductory course to practical aspects of digital systems design. Course includes the design of digital circuits with CAD tools and VHDL hardware description language.
		\end{smallQuote}
	}

	\entry
	{2012 -- 2014}
	{Aerospace Engineering Technician}
	{École Nationale d'Aérotechnique | QC, Canada}
	{Completed 56 credit towards a College diploma:
		\begin{smallQuote}
			Conceive and fabricate aeronautical parts using Computer Aided Design (CAD) tool---Catia V5;\\
			Plan fabrication step and assembly of airframe, engines, and other aeronautical components;\\
			Fabricate elements of the airframe using composite materials and sheet metal;\\
			Collaborate to the planification of the production process, the quality assurance and testing of aerospace components;\\
			Elaborate assembly booklets.
		\end{smallQuote}
	}

\end{entrylist}

\cvsect{Awards}

\begin{entrylist}
	\entry
	{2015 -- 2016}
	{Dean's List}
	{University of New Brunswick | Canada}
	{Awarded to students who maintained a GPA above 3.7 throughout the academic year.}
\end{entrylist}

\cvsect{Scholarships}

\begin{entrylist}
	\entry
	{2015 -- 2016}
	{Sir George E. Foster}
	{University of New Brunswick | Canada}
	{2000\$/yr. Awarded to students who maintained a GPA above 3.7 throughout the academic year.}

	\entry
	{2018 -- 2020}
	{Research Fellowship}
	{University of New Brunswick | Canada}
	{23,500\$/yr. research fellowship  with the Centre for Advanced Studies---Atlantic on the Eclipse OMR project}
\end{entrylist}

\cvsect{Experience}

\begin{entrylist}
	\entry{ Contracts\\
	{\footnotesize SM-2019}\\
	}
	{Instructor}
	{University of New Brunswick | Canada}{
	\textbullet{} CS2263---Systems Software Development
	\begin{quoting}
		Procedural program development and supporting tools, using the C language.
		Topics include implementation of data structures, memory management, compilation/linking, building, debugging, and version control.
	\end{quoting}
	}

	\entry{ Contracts\\
	{\footnotesize WI-2018}\\
	{\footnotesize SM-2018}\\
	{\footnotesize FA-2018}\\
	}
	{Teaching Assistant}
	{University of New Brunswick | Canada}
	{
	\textbullet{} Computability and Formal Languages\\
	\textbullet{} Professional Practices\\
	\textbullet{} Professional Practices\\
	}

	\entry{ Contracts\\
	{\footnotesize FA-2018}\\
	}
	{Tutor}
	{University of New Brunswick | Canada}{
	\textbullet{} Operating Systems I (CS3413) Rachelle Potter
	}

	\entry
	{2015 -- 2018}
	{Electronic and Software Technician}
	{Canada Electronic Parts Online Inc. | Canada}{
		Repairing electronic devices, i.e. smart watches, televisions, computers.
		\begin{tightemize}
			\item Played a significant role in increasing sales and revenue by increasing outreach.
			\item Broadened the company expertise in electronic repairs and introduced tooling and techniques to repair devices with surface mount components.
			\item Increased repairs success rate and revenue.
			\item Introduced software-based repairs and debugging (i.e. reflashing lost IMEI or CMOS)
			\item Introduced corrupted or defective hard drive data recovery.
		\end{tightemize}
	}

\end{entrylist}

\cvsect{Project Contributions}

\begin{entrylist}

	\entry
	{2018 -- cur.}
	{Eclipse OMR}
	{\href{https://github.com/eclipse/omr}{github://eclipse/omr}}{
		Responsible for initiating the work on the basic implementation of an ARM64V8 specific OMR; This involved making the full suite of hardware operation in the ARM64V8 documentation (or AArch64) available to the JIT compiler.
		Weekly meetings were held on infrastructure design and implementation.
		The project was lead by the IBM Runtimes Team in Ottawa, and work was done conjointly with members of the IBM Japan team.
	}

	\entry
	{2017 -- cur.}
	{Odin II}
	{\href{https://github.com/verilog-to-routing/vtr-verilog-to-routing/graphs/contributors}{github://verilog-to-routing}}{
		Maintain the Front-End Verilog Synthesis tool of the Verilog to Routing Project.
		\begin{tightemize}
			\item Managing other developers and review their merge requests.
			\item Fix Bugs and broken features.
			\item Build new features and improve the tool usability and documentation for both developers and users.
		\end{tightemize}
	}

	\entry
	{2019 \\{\footnotesize Feature contract}}
	{Butterfly Energy Systems}
	{\href{http://www.butterflyenergy.ca/home.php}{http://www.butterflyenergy.ca}}{
	Responsible to track down a null pointer exception in a multithreaded C++ codebase.
	Traced back to a race condition because of uncaught spurious wake of the threads.
	}

\end{entrylist}

\cvsect{Data Interchange Formats}

\bulletedlist{
	JSON,
	XML,
	TOML,
	YAML
}

\vspace*{\baselineskip}
\cvsect{Tools}

\bulletedlist{
	Make,
	CMake,
	Bison,
	Flex,
	Bash,
	Git,
	Jenkins,
	Docker,
	KVM,
	Cross-Compilers,
	Quartus,
	Verilog-To-Routing,
	GDB,
	Valgrind,
	AutoCAD,
	FreeCAD,
	Office Suite,
	CatiaV5
}

\vspace*{\baselineskip}
\cvsect{Operating Systems}

\bulletedlist{
	Linux,
	BSD,
	OSx,
	Windows
}

\vspace*{\baselineskip}
\cvsect{Other Knowledge}

\bulletedlist{
	Networking,
	Linux Development,
	Drafting,
	Sheet Metal Work,
	Welding,
	Electronic,
	Embedded System Development,
	CAD Drawing
}

\vspace*{\baselineskip}
\cvsect{Interest}

\bulletedlist{
	Electronics,
	Networking,
	Programming,
	Fabrication,
	Machining,
	Wood Working,
	Automation,
	Backcountry Camping,
	Hiking,
}

\ifdefined\academicCV {%

	\vspace*{\baselineskip}
	\cvsect{Travel Grant}

	\begin{entrylist}

		\shortentry{2020}{International Symposium on FPGA}{Monterey, California, United-States}
		\shortentry{2019}{Cascon}{Markham, Ontario, Canada}
		\shortentry{2018}{Cascon}{Markham, Ontario, Canada}
		\shortentry{2018}{Castle}{Markham, Ontario, Canada}
		\shortentry{2018}{Splash}{Boston, Massachusetts, Canada}
		\shortentry{2018}{CMC Microsystem TEXPO}{Toronto, Ontario, Canada}

	\end{entrylist}

	\vspace*{\baselineskip}
	\cvsect{Mentoring {\footnotesize(Unofficial Supervision)}}

	\begin{entrylist}

		\shortentry{2020 -- cur.}{Daniel Stokes}{Honours, B.CS., University of Waikato, New Zealand}
		\shortentry{2020 -- cur.}{Eve MacDonald}{Work Term, B.EE, University of New Brunswick, Canada}
		\shortentry{2020}{Shrey Vyas}{Work Term, M.CS, University of New Brunswick, Canada}
		\shortentry{2019 -- 2020}{Mohammad Sohrabi}{Ph.D. CS, University of New Brunswick, Canada}
		\shortentry{2019 -- cur.}{Seyed Alireza Damghani}{M.  CS, University of New Brunswick, Canada}
		\shortentry{2019}{Michael Flawn}{Honours, B.CS., University of New Brunswick, Canada}
		\shortentry{2019}{Alexandrea Demmings}{Work Term, B.SwE, University of New Brunswick, Canada}
		\shortentry{2019}{Julie Brown}{Work Term, B.CEE, University of New Brunswick, Canada}
		\shortentry{2018}{Nasrin Eshraghi Ivari}{Ph.D. CS University of New Brunswick, Canada}

	\end{entrylist}

} \fi

\ifdefined\academicCV {%

	\nocite{*}
	\blfootnote{Presentors are \underline{underlined}}

	\defbibheading{pip}{\vspace*{-6pt}\vspace*{\baselineskip}\cvsect{In Proceeding Papers}\vspace*{-6pt}}
	\printbibliography[heading=pip, keyword=inprogresspapers]

	\defbibheading{pubp}{\vspace*{-6pt}\vspace*{\baselineskip}\cvsect{Published Papers}\vspace*{-6pt}}
	\printbibliography[heading=pubp, keyword=publishedpapers]

	\defbibheading{diss}{\vspace*{-6pt}\vspace*{\baselineskip}\cvsect{Dissertations}\vspace*{-6pt}}
	\printbibliography[heading=diss, keyword=dissertations]

	\defbibheading{talks}{\vspace*{-6pt}\vspace*{\baselineskip}\cvsect{Talks}\vspace*{-6pt}}
	\printbibliography[heading=talks, keyword=talks]

	\defbibheading{poster}{\vspace*{-6pt}\vspace*{\baselineskip}\cvsect{Conference Posters}\vspace*{-6pt}}
	\printbibliography[heading=poster, keyword=poster]

} \fi

\ifdefined\academicCV {%
	\vspace*{-6pt}
	\vspace*{\baselineskip}
	\cvsect{Academic References}
} \else {%
	\clearpage
	\cvsect{References}
} \fi

\begin{entrylist}

	\entry%
	{Dr. Kenneth B. Kent}{Thesis Supervisor}
	{University of New Brunswick | Fredericton, New Brunswick, Canada}
	{\texttt{ken@unb.ca | +1 506 451 6971}\\
		Supervised me during my master's thesis, and is the code owner of the Odin II project.
	}

	\entry%
	{Dr. Paul Cook}{Assistant Dean}
	{University of New Brunswick | Fredericton, New Brunswick, Canada}
	{\texttt{paul.cook@unb.ca | +1 506 447 3466}\\
		Responsible for undergraduate affairs, he was my superior during my work as an instructor.
	}

	\entry%
	{Dr. Panos Patros}{Mentor and External Collaborator}
	{University of Waikato, Waikato, New Zealand}
	{\texttt{panos.patros@waikato.ac.nz | +64 7 838 4651 EXT:4651}\\
		Working under the CASA group, he taught me "Introduction to Java."
		He later mentored me during my honours thesis, which spun off a joint paper with him.
		I am currently mentoring one of his students in New Zealand.
	}

	\entry%
	{Aaron G. Graham}{Research Assistant}
	{University of New Brunswick, Fredericton, New Brunswick, Canada}
	{\texttt{aaron.graham@unb.ca | +1 506 453 3567}\\
		Responsible for managing day to day operations across projects in CASA, he is my superior on the Eclipse OMR project.
	}

\end{entrylist}

\ifdefined\coverLetter {%

	\clearpage

	\newgeometry{top=2.5cm, bottom=2.5cm, right=2.5cm, left=2.5cm}

	\cvsect{statement of purpose}

	\large
	\justifying
	\parskip 0ex plus 0.2ex minus 0.1ex
	\parindent1.5em
	\fontsize{12}{24}\selectfont

	I have worked on both commercial and open-source tools.
	My broad interest allows me to have an open mind about different topics and use lateral thinking to solve complex problems.
	I have done my undergraduate and master's at the University of New Brunswick, and I am now looking for new opportunities.

	During my undergraduate degree, I met a teacher that sparked my desire to do research.
	I tailored my undergraduate for this research and, two years later, applied to an honours with the Centre for Advanced Studies--Atlantic.
	The CASA research group offered me a scholarship for a master's on the Eclipse OMR project.
	Throughout my Masters, I became the maintainer of the Odin II project and co-authored five papers on the subject.
	I became responsible for training summer, honours, masters, and Ph.D. students for Odin II.
	During the summer of 2019, an opportunity opened up to teach, and I had three weeks to prepare.
	I was responsible for teaching 70 students about C for their first time;
	Within six weeks, I taught them memory management (stack vs heap memory), git, Makefiles, Bison, and Flex.
	I loved every bit of it: the long nights preparing and reviewing material, and the open office door to help students.

	I have grown significantly at the CASA group but have started to stagnate.
	In applying to do my Ph.D. under you, I wish to increase my breadth of knowledge and get new opportunities to teach and publish.
	I am grateful that you took the time to read this and am looking forward to hearing from you.\\

	\noindent Sincerely,\\
	\indent Jean-Philippe Legault

} \fi

\end{document}

