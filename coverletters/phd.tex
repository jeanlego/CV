I have worked on both commercial and open-source tools.
My broad interests allow me to have an open mind about different topics and use lateral thinking to solve complex problems.
I have done my undergraduate and Master’s at the University of New Brunswick, and I am now searching for new opportunities.

I acquired three years of studies in Aerospace Engineering at a college, lending me a solid background in mission-critical systems.
During my studies, I have acquired strong understanding and abilities in mechanical engineering concepts such as CAD, assembly, metallurgy, and composite materials;
My first introduction to CAD tools quickly made me realize the potential and the drawbacks of the tools available; I changed vocation to computer science to learn to improve these.

My rapid growth from having no experience in software development when I started my bachelor’s (2015) is a testament to my desire and ability to learn and create.
During my undergraduate , I joined my studies with my passion for electronics. My Honour’s topic was on FPGA architecture development using the Odin II CAD tool—an FPGA computer-aided design, Verilog synthesis tool.
I have continued to work on the project and am currently the lead developer; this has allowed me to learn to design, maintain, and administer projects.
I have contributed to numerous papers, mentored, and trained fellow researchers from Canada to New Zealand.
The experience I have acquired working under IBM for my Master’s has allowed me to learn to develop features for large-scale projects and collaborate with team members worldwide.
My experience with digital compilers (Verilog synthesis tool) and JIT compilers (JVM) give me a profound understanding of how compilers and high-level languages operate.

During my masters, I taught system software development, a C course touching on version control and build systems.
Within six weeks, I taught them memory management (stack vs. heap memory), git, Makefiles, Bison, Flex, and introduced them to C.
I loved it: the long nights preparing material and the open office to help students.

Outside of computer science, electronics has always been a passion; I have been “modding” gaming consoles ever since I learned to solder in elementary school.
I have professionally repaired household devices like power amplifiers, smartphones, televisions, computers, motherboards.
I have experience working with surface mount components and understand the challenges they pose.

I can be an asset since I can easily mentor and learn, remotely or on-site, as shown by my experience.
I am frank, loyal, and autonomous; I will take on many responsibilities as I strive to learn, share, and create.
